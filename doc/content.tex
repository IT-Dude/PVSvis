\section{Allgemeines}
	\begin{itemize}
		\item Praktikumsleiter Steffen Dienst (\url{dienst@informatik.uni-leipzig.de})
		\item Git: \url{www.github.com/IT-Dude/PVSvis}
	\end{itemize}

\section{Aufgabe}
	\begin{itemize}
		\item Visualisierung einer gro�en Menge durch Solaranlagen
		erzeugte Wechselrichterdaten
		\item Evaluierung mehrerer Javascript-Bibliotheken zur Visualisierung
		\item interaktive Darstellung im Browser
		\item Must-haves der Anwendung:
			\begin{itemize}
  				\item Anzeige verschiedener Diagrammtypen (Linien-, Scatter-,
  				Balkendiagramme \dots)
  				\item Verlinkungen zwischen unterschiedlichen Diagrammen
  				\item Darstellung mehrerer Achsen innerhalb eines Diagramms
  				\item Tooltips an Messpunkten \dots
  				\item Hilfslinien, Lineale \dots
  				\item Markierung von Werten
  				\item Ein-/Ausblenden von Messwerten und -reihen
			\end{itemize}
	\end{itemize}

\section{�berblick �ber mehrere Bibliotheken}
	\subsection{CanvasXpress}
		\url{www.canvasxpress.org}
		\begin{itemize}
			\item Grafiken werden im HTML5 Canvas-Element erstellt
			\item Balken-, Linien-, Kerzen-, Scatter-, Kuchendiagramme, Heatmaps \dots
			\item gro�e Anzahl verschiedener Diagrammtypen
			\item viele Konfigurationsm�glichkeiten
			\item Unterst�tzung von JSON
			\item verschiedene Diagramme ben�tigen bestimmte Struktur der Daten (als
			JSON)
			\item Erstellung eines Diagramms aus Daten, Konfiguration und Events
			\item absolute High-Level-Bibliothek
			\item Zeichnen einfacher Formem m�glich
			\item Dokumentation ist der hohen Abstraktion angemessen
			\item GNU LGPL
		\end{itemize}
	
	\subsection{Cube}
		\url{www.square.github.com/cube/}
		\begin{itemize}
			\item nutzt unter anderem D3, Node.js und MongoDB
			\item besitzt eine serverseitige Komponente (Collector)
			\item speziell zur Analyse von Zeitreihen entworfen
			\item Anforderung von Daten �ber das WebSocket-Protokoll
			\item Zusammenstellen von Diagrammansichten �ber eine interaktive Oberfl�che
			\item eigene kleine Sprache f�r Anfragen, Filter
			\item Cube empf�ngt Events, fertigt Metriken anhand von Anfragen �ber diese
			Events an
			\item Metriken werden zwischengespeichert
			\item Visualisierung der Anfragen clientseitig
			\item ``Emitters'' senden Daten an Cubes ``Collectors''
			\item einige wenige Grundvisualisierungen vorhanden
			\item Erstellung eigener Visualisierungen m�glich (wahrscheinlich mit D3)
			\item Dokumentation sehr sp�rlich, fast nicht vorhanden
			\item Apache License
		\end{itemize}

	\subsection{D3}
		\url{www.mbostock.github.com/d3}
	
	\subsection{Dygraphs}
		\url{www.dygraphs.com}
		\begin{itemize}
			\item auv Visualisierung von Zeitreihen zugeschnitten
			\item interaktive, zoombare Diagramme
			\item Visualisierung im Canvas-Element
			\item Unterst�tzung von CSV-Dateien
			\item Darstellung automatisiert (Farbwahl, Aufl�sung der Skalen \dots)
			\item kompatibel zur Google Visialization API (Datenformat)
			\item auf eine Liniendiagramm-Art beschr�nkt
			\item hoher Abstraktionsgrad
			\item einige Konfigurationsm�glichkeiten
			\item sehr kleine Dokumentation
			\item MIT-Lizenz
		\end{itemize}
	
	\subsection{Elycharts}
		\url{www.elycharts.com}
		\begin{itemize}
			\item nutzt jQuery und Rapha�l
			\item mehrere Diagrammtypen
			\item mehrere Datens�tze und Achsen innerhalb eines Diagramms
			\item viele optische Konfigurationsm�glichkeiten
			\item Interaktion mit Diagrammen
			\item Diagramme werden als SVG dargestellt
			\item optionale Templates f�r Diagramme
			\item High-Level-Bibliothek
			\item MIT-Lizenz
		\end{itemize}

	\subsection{Google Chart Tools}
		\url{www.code.google.com/intl/de-DE/apis/chart/}
		\begin{itemize}
			\item Grafikdarstellung als SVG
			\item viele vordefinierte Diagrammtypen
			\item sehr viele Konfigurationsm�glichkeiten
			\item interaktive Diagramme, Events
			\item High-Level-Bibliothek, sehr sehr viele Funktionen
			\item eigene Sprache f�r Anfragen, �hnlich SQL
			\item Unterst�tzung von JSON, CSV, TSV (tab-seperated values) und HTML als
			Datenformate
			\item weitreichende Dokumentation
			\item Google-eigene Lizenz
		\end{itemize}

	\subsection{Highcharts}
		\url{www.highcharts.com/products/highcharts}
		\begin{itemize}
			\item Grafiken als SVG
			\item viele Diagrammtypen, vereinbar innerhalb eines Diagramms
			\item interaktive Diagramme
			\item Unterst�tzung mehrerer Skalen und Achsen
			\item zoombare Diagramme
			\item Master-Detail-Chart �hnlich Dycharts
			\item Datenformate: JSON, CSV, XML
			\item hoher Abstraktionsgrad; Zugriff auf niedere, einfache
			Zeichenfunktionen
			\item ausreichende Dokumentation
			\item kommerzielle Lizenz, Creative Commons Lizenz f�r nichtkommerzielle
			Projekt
		\end{itemize}
	
	\subsection{InfoVis}
		\url{www.thejit.org}
		\begin{itemize}
			\item viele, komplexere Diagrammtypen (TreeMap, SpaceTree, RGraph)
			\item erh�hter Grad an Animation und Interaktion, abh�ngig vom Diagrammtyp
			\item gute Dokumentation
			\item JSON nutzbar
			\item High-Level-Bibliothek
			\item wenig allgemeine Informationen
			\item sehr aktive Entwicklung auf GitHub
		\end{itemize}
	
	\subsection{Protovis}
		\url{www.mbostock.github.com/protovis}
		\begin{itemize}
			\item wird seit Juni 2011 nicht mehr weiterentwickelt
			\item Inspiration zu D3, von den gleichen Entwicklern
			\item nutzt SVG
			\item Binden von Daten and einfache grafische Elemente, direkte grafische
			Repr�sentation eines Datensatzes
			\item Breitstellung von Grundelementen, Erweiterung und Vererbung dieser
			Elemente
			\item einige grundlegende Diagrammtypen
			\item Erstellung eigener, komplexer Visualisierungen
			\item Erstellung der Visualisierung anhand der Daten
			\item viel Interaktion
			\item hohe Komplexit�t von Visualisierungen und Animationen
			\item sehr viele M�glichkeiten der Konfiguration
			\item Medium-Level-Bibliothek
			\item sehr gute Dokumentation, viele Beispiele
			\item BSD Lizenz
		\end{itemize}
	
	\subsection{Andere Bibliotheken}
		\texttt{Diese Bibliotheken sind sehr primitiv gehalten beziehungsweise
		beschr�nken sich auf Low-Level-Grafikerstellung.}
		\begin{itemize}
			\item Flot (\url{www.code.google.com/p/flot})
			\item Rapha�l (\url{www.raphaeljs.com})
			\item Processing.js (\url{www.processingjs.org})
		\end{itemize}

\section{Entscheidung f�r eine Bibliothek}
\todo{Entscheidung vern�nftig begr�nden!}

\section{Implementierung}
	\begin{itemize}
		\item interessantes Konzep bei Dycharts
	\end{itemize}

\section{Treffen}
	\subsection{20.10.2011}
		\begin{itemize}
			\item Besprechung der Aufgabenstellung
			\item Festlegung der Must-haves
		\end{itemize}

\section{Ideen}
	\begin{itemize}
		\item dunkler Hintergrund, helle Visualisierung \textrightarrow mehr Energie
		\item Sonne nachahmen ???
	\end{itemize}

\section{Notizen}
	\begin{itemize}
		\item Visualisierung gro�er Zeitr�ume, geeignete Methoden?
		\item \url{bl.ocks.org}
		\item \url{jsfiddle.net}
		\item Github Pages
		\item Processing
		\item
		\url{www.sixrevisions.com/javascript/20-fresh-javascript-data-visualization-libraries}
		\item
		\url{www.splashnology.com/article/15-awesome-free-javascript-charts/325/}
		\item
		\url{www.flowingdata.com/2011/03/09/data-driven-documents-for-visualization-in-the-browser}
		\item
		\url{www.flowingdata.com/2010/01/07/11-ways-to-visualize-changes-over-time-a-guide}
		\item \url{www.dekstop.de/weblog/2011/09/lastfm_heatmap_calendars}
		\item Heatmap Kalender
		\item \url{www.datavis.dekstop.de/last.fm_heatmap_calendars}
	\end{itemize}