\section{Allgemeines}
	\begin{itemize}
		\item Praktikumsleiter Steffen Dienst (\url{dienst@informatik.uni-leipzig.de})
		\item Git: \url{www.github.com/IT-Dude/PVSvis}
	\end{itemize}

\section{Aufgabe}
	\begin{itemize}
		\item Visualisierung einer gro�en Menge durch Solaranlagen
		erzeugte Wechselrichterdaten
		\item Evaluierung mehrerer Javascript-Bibliotheken zur Visualisierung
		\item interaktive Darstellung im Browser
		\item Must-haves der Anwendung:
			\begin{itemize}
  				\item Anzeige verschiedener Diagrammtypen (Linien-, Scatter-,
  				Balkendiagramme \dots)
  				\item Verlinkungen zwischen unterschiedlichen Diagrammen
  				\item Darstellung mehrerer Achsen innerhalb eines Diagramms
  				\item Tooltips an Messpunkten \dots
  				\item Hilfslinien, Lineale \dots
  				\item Markierung von Werten
  				\item Ein-/Ausblenden von Messwerten und -reihen
			\end{itemize}
	\end{itemize}

\section{�berblick �ber mehrere Bibliotheken}
	Kriterien: Vorlagen, Komplexit�t, Dokumentation, Leistungsf�higkeit,
	Interaktivit�t, Design, Kontrolle, Technik, Lizenz, Low- / High-Level \dots
	
	\subsection{CanvasXpress}
		\url{www.canvasxpress.org}
		\begin{itemize}
			\item Grafiken werden im HTML5 Canvas-Element erstellt
			\item Balken-, Linien-, Kerzen-, Scatter-, Kuchendiagramme, Heatmaps \dots
			\item gro�e Anzahl verschiedener Diagrammtypen
			\item viele Konfigurationsm�glichkeiten
			\item Unterst�tzung von JSON
			\item verschiedene Diagramme ben�tigen bestimmte Struktur der Daten (als
			JSON)
			\item Erstellung eines Diagramms aus Daten, Konfiguration und Events
			\item absolute High-Level-Bibliothek
			\item Zeichnen einfacher Formem m�glich
			\item Dokumentation ist der hohen Abstraktion angemessen
			\item GNU LGPL
		\end{itemize}
	
	\subsection{Cube}
		\url{www.square.github.com/cube/}
		\begin{itemize}
			\item nutzt unter anderem D3, Node.js und MongoDB
			\item besitzt eine serverseitige Komponente (Collector)
			\item speziell zur Analyse von Zeitreihen entworfen
			\item Anforderung von Daten �ber das WebSocket-Protokoll
			\item Zusammenstellen von Diagrammansichten �ber eine interaktive Oberfl�che
			\item eigene kleine Sprache f�r Anfragen, Filter
			\item Cube empf�ngt Events, fertigt Metriken anhand von Anfragen �ber diese
			Events an
			\item Metriken werden zwischengespeichert
			\item Visualisierung der Anfragen clientseitig
			\item ``Emitters'' senden Daten an Cubes ``Collectors''
			\item einige wenige Grundvisualisierungen vorhanden
			\item Erstellung eigener Visualisierungen m�glich (wahrscheinlich mit D3)
			\item Dokumentation sehr sp�rlich, fast nicht vorhanden
			\item Apache License
		\end{itemize}

	\subsection{D3}
		\url{www.mbostock.github.com/d3}
	
	\subsection{Dygraphs}
		\url{www.dygraphs.com}
		\begin{itemize}
			\item auv Visualisierung von Zeitreihen zugeschnitten
			\item interaktive, zoombare Diagramme
			\item Visualisierung im Canvas-Element
			\item Unterst�tzung von CSV-Dateien
			\item Darstellung automatisiert (Farbwahl, Aufl�sung der Skalen \dots)
			\item kompatibel zur Google Visialization API (Datenformat)
			\item auf eine Liniendiagramm-Art beschr�nkt
			\item hoher Abstraktionsgrad
			\item einige Konfigurationsm�glichkeiten
			\item sehr kleine Dokumentation
			\item MIT-Lizenz
		\end{itemize}
	
	\subsection{Elycharts}
		\url{www.elycharts.com}
		\begin{itemize}
			\item nutzt jQuery und Rapha�l
			\item mehrere Diagrammtypen
			\item mehrere Datens�tze und Achsen innerhalb eines Diagramms
			\item viele Konfigurationsm�glichkeiten
			\item Interaktion mit Diagrammen
			\item Diagramme werden als SVG dargestellt
			\item optionale Templates f�r Diagramme
			\item High-Level-Bibliothek
			\item MIT-Lizenz
		\end{itemize}

	\subsection{Google Chart Tools}
		\url{www.code.google.com/intl/de-DE/apis/chart/}

	\subsection{Highcharts}
		\url{www.highcharts.com/products/highcharts}
	
	\subsection{InfoVis}
		\url{www.thejit.org}
	
	\subsection{Protovis}
		\url{www.mbostock.github.com/protovis}
	
	\subsection{Andere Bibliotheken}
		\begin{itemize}
			\item Flot (\url{www.code.google.com/p/flot})
			\item Rapha�l (\url{www.raphaeljs.com})
			\item Processing.js (\url{www.processingjs.org})
		\end{itemize}

\section{Entscheidung f�r eine Bibliothek}
\todo{Entscheidung vern�nftig begr�nden!}

\section{Implementierung}
	\begin{itemize}
		\item interessantes Konzep bei Dycharts
	\end{itemize}

\section{Treffen}
	\subsection{20.10.2011}
		\begin{itemize}
			\item Besprechung der Aufgabenstellung
			\item Festlegung der Must-haves
		\end{itemize}

\section{Ideen}
	\begin{itemize}
		\item dunkler Hintergrund, helle Visualisierung \textrightarrow mehr Energie
		\item Sonne nachahmen ???
	\end{itemize}

\section{Notizen}
	\begin{itemize}
		\item Visualisierung gro�er Zeitr�ume, geeignete Methoden?
		\item \url{bl.ocks.org}
		\item \url{jsfiddle.net}
		\item Github Pages
		\item Processing
		\item
		\url{www.sixrevisions.com/javascript/20-fresh-javascript-data-visualization-libraries}
		\item
		\url{www.splashnology.com/article/15-awesome-free-javascript-charts/325/}
		\item
		\url{www.flowingdata.com/2011/03/09/data-driven-documents-for-visualization-in-the-browser}
		\item
		\url{www.flowingdata.com/2010/01/07/11-ways-to-visualize-changes-over-time-a-guide}
		\item \url{www.dekstop.de/weblog/2011/09/lastfm_heatmap_calendars}
		\item Heatmap Kalender
		\item \url{www.datavis.dekstop.de/last.fm_heatmap_calendars}
	\end{itemize}